\documentclass{beamer}

\mode<presentation>{
\usetheme{Dresden}
\setbeamercovered{transparent}
\usecolortheme{lsc}
}

\mode<handout>{
  % tema simples para ser impresso
  \usepackage[bar]{beamerthemetree}
  % Colocando um fundo cinza quando for gerar transparências para serem impressas
  % mais de uma transparência por página
  \beamertemplatesolidbackgroundcolor{black!5}
}

\usepackage{amsmath,amssymb}
\usepackage[brazil]{varioref}
\usepackage[english,brazil]{babel}
\usepackage[utf8]{inputenc}
%\usepackage[latin1]{inputenc}
\usepackage{graphicx}
\usepackage{listings}
\usepackage{url}
\usepackage{colortbl}
\usepackage[ruled, linesnumbered]{algorithm2e}
\usepackage{amsmath}

\newcommand\Fontvi{\fontsize{6}{10}\selectfont}

\beamertemplatetransparentcovereddynamic

\title[Machine Learning for Probabilistic Robotics with Webots]{Machine Learning for Probabilistic Robotics with Webots}
\author[Joan Gerard]{%
  Joan Gerard\inst{1} \\
  Promotor: Prof. Gianluca Bontempi \inst{1}}
  \institute[ULB]{
  \inst{1}%
     Universit\'e Libre de Bruxelles}

% Se comentar a linha abaixo, irá aparecer a data quando foi compilada a apresentação  
\date{June 7, 2019}



\AtBeginSection[]{
  \begin{frame}<beamer>
    \frametitle{Table of Contents}
    \tableofcontents[currentsection,currentsubsection]
  \end{frame}
}

\begin{document}

\begin{frame}
\titlepage
\end{frame}

\begin{frame}
\frametitle{Table of Contents}
\tableofcontents
\end{frame}

\section{Probabilistic Robotics}
\pgfdeclareimage[height=6cm]{RM}{figs/whereAmI.png}
\frame{
    \frametitle{Where am I?}
            \begin{itemize}
                \item Robot knows where it is
                \item Robot is kidnapped
                \item Robot does not know where it is
            \end{itemize}
            \pgfuseimage{RM}

}

\pgfdeclareimage[height=6cm]{BAYES-FILTER-1}{figs/bayes-filter-1.png}%
\pgfdeclareimage[height=6cm]{BAYES-FILTER-2}{figs/bayes-filter-2.png}%
\frame{
	\frametitle{Let's figure it out using Bayes Filter!}
	\begin{columns}
	    \column{5.5cm}
	        \begin{itemize}
	            \item Prediction
	        \end{itemize}
        	\pgfuseimage{BAYES-FILTER-2}
        
        \column{5.5cm}
            \begin{itemize}
	            \item Error Correction
	        \end{itemize}
        	\pgfuseimage{BAYES-FILTER-1}
	\end{columns}
}

\pgfdeclareimage[height=3cm]{PARTICLE-FILTER-1}{figs/particles-1.png}%
\pgfdeclareimage[height=3cm]{PARTICLE-FILTER-2}{figs/particles-2.png}%
\pgfdeclareimage[height=4.0cm]{PARTICLE-FILTER-3}{figs/particles-3.png}%
\frame{
	\frametitle{What about Particles Filter?}
	\centering

    	\pgfuseimage{PARTICLE-FILTER-1}
    	\pgfuseimage{PARTICLE-FILTER-2}\\
    	\pgfuseimage{PARTICLE-FILTER-3}
}

\begin{frame}
	\frametitle{Particles Filter Algorithm}

	\begin{itemize}
		\item Train/test a prediction model: robot state $\rightarrow$ distance sensor data
		\item Compare it with the true sensor data and determine a weight for each particle
		\item More close to the real sensor data more weight will have
	\end{itemize}
		\Fontvi
		\begin{columns}
			\column{9cm}
			\begin{algorithm}[H]

\SetKwInOut{Input}{input}\SetKwInOut{Output}{output}
\Input{particles, controlAction, sensorData}
\Output{nextParticles}
$nextParticles \leftarrow \emptyset$\\
\BlankLine
  \ForEach{particle $\in$ particles}{
    particle.state $\leftarrow$ predictState(particle.state, controlAction) \\
    particle.weight $\leftarrow$ calculateWeight(particle.state, sensorData)\\
  }
  \For{$m=1 \text{ to } m=|partices|$}{
     $newParticle \leftarrow \text{draw } i \text{ from } particles \text{ with probability } \propto particles[i].weight$\\
     nextParticles.add(newParticle)\\
  }
  \Return nextParticles
\end{algorithm}

		\end{columns}
		
\end{frame}

\section{Webots}

\pgfdeclareimage[height=3cm]{WEBOTS}{figs/webots.png}%
\pgfdeclareimage[height=3cm]{ROBOT}{figs/robot.png}%
\pgfdeclareimage[height=3cm]{WORLD}{figs/world.png}%
\frame{
	\begin{columns}
    \frametitle{What tool can we use for simulating that problem?}
        \column{4cm}
            Webots!
            \begin{itemize}
            		\item Open Source
	                \item Python, C++, Java, etc.
	                \item 44 robot models
	                \item Robot model creation/customization
	                \item Robust documentation
            \end{itemize}
            
            
		\column{5cm}
			\pgfuseimage{WEBOTS}\\
			\pgfuseimage{ROBOT}
			\pgfuseimage{WORLD}
			
    \end{columns}
}

\section{Project Objectives}
\frame{
	\frametitle{Main Objective}
	\begin{itemize}
	    \item The objective will be to exploit the simulation benefits of Webots to introduce Machine Learning techniques together with non-parametric filters (such as Particles Filter) for robot positioning estimation, independently from the kind of robot used for in-door environments.
	\end{itemize}
}

\frame{
    \frametitle{How do we plan to do it?}
    \begin{itemize}
        \item Using Webots
        \begin{itemize}
            \item Create a custom robot
            \item Randomize control actions while moving avoiding obstacles
	        \item Use a realistic in-door environment
        \end{itemize}{}
	    \item Using Machine Learning and Particles Filters
	        \begin{itemize}
                \item Positioning estimation
    	        \item Error correction
        \end{itemize}
	    \item Test the developed technique with other robots
	\end{itemize}
}


\section{Preliminary Results}
\pgfdeclareimage[height=3.8cm]{RESULTS}{figs/position3.png}%
\pgfdeclareimage[height=3.5cm]{RESULTS1}{figs/arena.png}%
\frame{
	\frametitle{What have we done so far?}
	
	\begin{itemize}
	    \item Robot creation
	    \item Dummy environment creation
	    \item Collect distance sensor data + real position coordinates
	    \item Bayes Filter algorithm implementation
    	    \begin{itemize}
    	        \item \textit{Position prediction}: Odometry techniques
    	        \item \textit{Error correction}: Training a Random Forest model using collected data
    	    \end{itemize}
	\end{itemize}
	
	\pgfuseimage{RESULTS1}
	\pgfuseimage{RESULTS}
}

\frame{

    \frametitle{Are the results good?}
    \begin{itemize}
        \item Yes! but there are improvements to be done:
        \begin{itemize}
            \item Collect more data (+ Simulations)
            \item Randomize robot moves 
            \item Train/test with other Machine Learning models
            \item Improve performance of the error correction algorithm
        \end{itemize}
    \end{itemize}
}

\pgfdeclareimage[height=6cm]{QUESTIONS}{figs/questions.png}%
\frame{
    \centering
    \pgfuseimage{QUESTIONS}
}


\end{document}
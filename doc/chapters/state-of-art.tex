\section{State of the art}
\subsection{Webots}
Webots was created by Cyberbotics Ltd. a spin-off company from the EPFL and has been developing it since 1998. It currently\footnote{2019} employs 6 people in Lausanne, Switzerland to continuously develop Webots according to customers needs. Cyberbotics provides consulting on both industrial and academic research projects and delivers open-source software solutions to its customers. It also provides user support and training to the users of the Webots software. The source code and binary packages are available for free; however, user support and consultancy are not. It is available for Windows, Ubuntu Linux and MacOS \cite{cyberbotics}.

Webots website has a reference manual that describes nodes and API functions. It has a complete user guide as well endowed with examples of simulations that show the use of actuators, creation of different environments, geometries primitives, complex behaviors, functionalities and advanced 3D rendering capabilities. This section is oriented to describe the basics of Webots and it is focused on what will be useful to develop the project. For a detailed description please refer to the user guide\footnote{\url{https://cyberbotics.com/doc/guide/index}} or the reference manual\footnote{\url{https://cyberbotics.com/doc/reference/index}}.

\subsubsection{Graphic Interface}
Webots supports the following programming languages: C, C++, Python, Java, MATLAB and ROS. Additionally, it offers the possibility of creating a custom interface to third-party software such as Lisp\textsuperscript{TM} or LabView\textsuperscript{TM} using TCP/IP protocol.

\begin{figure}[h]
	\includegraphics[width=\linewidth]{\images/chapter2/interface.png}
	\caption{Webots graphic interface}
	\label{fig-ch-2:interface}
\end{figure}

Figure \ref{fig-ch-2:interface} shows the main graphic interface of Webots. It can be divided in 5 panels:

\begin{enumerate}
	\item Simulation: graphic visualization of the simulated world objects.
	\item Code visualization: robot controller code editor.
	\item World Information: information about the simulated world.
	\item Console: program execution output stream.
	\item Control panel: set of buttons that controls the simulation execution.
\end{enumerate}

The program allows users to create highly personalized simulated environments which are called worlds, from scratch using pre-built 3D object models such as robots, wood boxes, walls, arenas, etc. A robot needs to be associated with a controller program that contains the source code with the desired behavior. This controller can be easily edited in the code panel and once it is saved it is automatically reloaded into all the robots associated with it. For running the simulation users can play, stop or reset it, among other options, using the control panel set of buttons. The output stream of the controller execution will be displayed in the console panel. 


\subsubsection{Webots with Python 3.7 and TensorFlow}

Python was created in 1990 by Guido van Rossum at Stichting Mathematisch Centrum in the Netherlands as a successor of a language called ABC\cite{python-docs}. Nowadays it has became one of the most popular programming languages for data science\cite{Raschka:2015:PML:2886323}.

The Python API of Webots was created from C++ API and it supports Python 3.7. It is possible to configure Webots to use Python 3.7 which should be previously installed from the Python website\footnote{Download it from \url{https://www.python.org}}; however, Webots does not work properly with Python versions installed from package managers as Brew. By default Webots is configured to use the default installed version of Python. In order to use another version, access to the menu options in Webots: \path{Webots/Preferences}; the \textit{Python command} label should point out to the installation path of Python3.7 as it is shown in figure \ref{fig-ch-2:python}. 

\begin{figure}[ht]
	\includegraphics[width=\linewidth]{\images/chapter2/preferences.png}
	\caption{Configure Python3.7}
	\label{fig-ch-2:python}
\end{figure}

Python allows to install third-party libraries like TensorFlow which is a Python-friendly open-source library developed by the researchers and engineers of the Google Brain team for internal use only and  then released in November 2015 under a permissive open source license. It implements machine learning algorithms and deep learning wrappers\cite{Raschka:2015:PML:2886323}.

The \verb|pip3| command allows to install any library for Python 3.7. For installing TensorFlow type \verb|pip3 install tensorflow| in the console terminal. For verifying the correct installation, the code displayed in listing \ref{code:verify-tf} can be put inside a controller application in Webots.

\begin{lstlisting}[language=Python, caption=Verify correct installation of TensorFlow, label=code:verify-tf]
import tensorflow as tf

verifier = tf.constant('TensorFlow was installed correctly.')
sess = tf.Session()
print(sess.run(verifier))
\end{lstlisting}

If TensorFlow was correctly installed, after the execution of the simulation, a message in the console panel will be displayed informing about its correct installation.


\subsubsection{Robots and world creation}

Webots allows creating large simulated worlds. The world description and content is presented as a tree structure where each node represents an object in the world, those objects have themselves nodes and sub-nodes within a name and a value indicating different physical characteristics or components. 

\begin{figure}[h!]
  \centering
  \includegraphics[width=0.8\linewidth]{\images/chapter2/nodes.png}
  \caption{World structure}
  \label{fig:ch-2:nodes}
\end{figure}

For adding a node into the world we can press the \verb|Add object| button that will open the window shown in figure \ref{fig:ch-2:nodes}. The base nodes are the basic objects that can be part of the world. Moreover, they are simpler models than the others. The USE nodes are objects that were already created in the world and can be reused. For instance, if a wood box was added into the world with some specific physical characteristics, a name can be assigned to its USE attribute and then it can be reused instead of creating a new one with same characteristics. The PROTO nodes contains a set of 3D models as robots, objects, vehicles, etc. Fruits, toys, drinks, plants, chairs, stairs and buildings are only a small part of the big set of modeled objects. The robots node offers as well a big diversity of models. From simple robots as the e-puck (figure \ref{fig:ch-2:e-puck}) model which is used very often in research, to more complex robots as the very well known humanoid Nao (figure \ref{fig:ch-2:nao}) are some examples of a total of 44 3D modeled robots that are available to use within the simulator tool. 

\begin{figure}[h!]
  \centering
  \begin{subfigure}[b]{0.3\linewidth}
  	\includegraphics[width=\linewidth]{\images/chapter2/e-puck.png}
  	\caption{e-puck}
  	\label{fig:ch-2:e-puck}
  \end{subfigure}
  \begin{subfigure}[b]{0.35\linewidth}
  	\includegraphics[width=\linewidth]{\images/chapter2/nao.png}
  	\caption{Nao}
  	\label{fig:ch-2:nao}
  \end{subfigure}
  \caption{Robots}
\end{figure}

Another powerful feature of Webots is to create a custom robot model from scratch using a tree-based structure of solid nodes which are virtual objects with physical properties. Thus a robot is made of a set of solid nodes put them together using joint nodes which are abstract nodes that models mechanical joints. Figure \ref{fig:ch-2:joints} shows the different types of joints that can be used.

\begin{figure}[h!]
  \centering
  \begin{subfigure}[b]{0.47\linewidth}
  	\includegraphics[width=\linewidth]{\images/chapter2/ballJoint.png}
  	\caption{Ball Joint}
  	\label{fig:ch-2:ball-joint}
  \end{subfigure}
  \vspace{0.00mm}
  \begin{subfigure}[b]{0.47\linewidth}
  	\includegraphics[width=\linewidth]{\images/chapter2/hinge2Joint.png}
  	\caption{Hinge 2 Joint}
  	\label{fig:ch-2:highe-2-joint}
  \end{subfigure}
  \vspace{0.00mm}
  \begin{subfigure}[b]{0.47\linewidth}
  	\includegraphics[width=\linewidth]{\images/chapter2/hingeJoint.png}
  	\caption{Hinge Joint}
  	\label{fig:ch-2:hinge-joint}
  \end{subfigure}
  \vspace{0.00mm}
  \begin{subfigure}[b]{0.47\linewidth}
  	\includegraphics[width=\linewidth]{\images/chapter2/sliderJoint.png}
  	\caption{Slider Joint}
  	\label{fig:ch-2:slider-joint}
  \end{subfigure}
  \caption{Different types of joint nodes}
  \source{Webots documentation}
  \label{fig:ch-2:joints}
\end{figure}


A position sensor can be added into the devices property of a joint node in order to monitor it. In like manner a rotational motor node can be added for actuating it. Thus, putting all together a simplistic version of a custom robot can be created based on the tree information and properties given to the simulator.







